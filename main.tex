\documentclass[12pt]{article}
\usepackage[T2A]{fontenc}
\usepackage[utf8]{inputenc}
\usepackage[russian]{babel}
\usepackage{tikz}
\usepackage{graphicx}
\usepackage{amsmath, amssymb, amsthm}

\usepackage[a4paper,margin=2.5cm]{geometry}
\usepackage{hyperref}

\usepackage{listings}
\lstset{
  basicstyle=\ttfamily\small,
  breaklines=true,
  columns=fullflexible,
  frame=single
}

\newtheorem{theorem}{Теорема}
\newtheorem{lemma}{Лемма}
\newtheorem{definition}{Определение}
\newtheorem{proposition}{Утверждение}
\newtheorem{corollary}{Следствие}

\newcommand{\VCk}{\mathrm{VC}_k}
\newcommand{\AVCk}{\mathrm{AVC}_k}
\newcommand{\VCFk}{\mathrm{VCF}_k}
\newcommand{\CVCFk}{\mathrm{cVCF}_k}
\newcommand{\AVCFk}{\mathrm{AVCF}_k}

\title{Запрещённые миноры для свойства Almost Vertex Covered}
\author{Шарыпов Егор}

\begin{document}
\maketitle

\begin{abstract}

Задача о вершинном покрытии является одной из классических $\mathbf{NP}$-полных задач, однако при фиксированном размере покрытия существует конечный список запрещённых миноров, отсутствием которых в графе определяется это свойство. В данной работе мы найдём списки запрещённых миноров для размеров покрытия от 1 до 6. Впервые список запрещённых миноров будет найден для следующей модификации этого свойства: при удалении любого ребра существует вершинное покрытие размера $k$, где $k$ также от 1 до 6.

\end{abstract}

\section{Введение}

Будем называть граф $H$ минором графа $G$ (и писать $H \le_m G$), если $H$ можно получить из $G$ цепочкой (возможно пустой) из следующих действий:

\begin{enumerate}
\item Удаление ребра
\item Удаление вершины
\item Стягивание ребра
\end{enumerate}

Будем называть свойство графов $\mathcal{A}$ замкнутым относительно взятия миноров, если $\forall G \in \mathcal{A} \forall H(H \le_m G \Rightarrow H\in\mathcal{A})$.

Теорема Робертсона-Сеймура \cite{robertson-seymour} гласит, что для любого замкнутого относительно взятия миноров свойства существует конечный список запрещённых миноров, отсутствие которых в графе равносильно наличию этого свойства. Причём эти запрещённые миноры -- это в точности те графы, которые сами не обладают заданным свойстом, но любой их минор будет обладать этим свойством.

К сожалению, доказательство теоремы не конструктивно, и поэтому списки запрещённых миноров для многих свойств остаются открытыми проблемами.

В этой работе нас будут интересовать свойства, связанные с вершинным покрытием. Вершинное покрытие графа -- подмножество вершин, такое, что любое ребро графа инцидентно некоторой вершине покрытия.

Введём обозначения:

$\tau(G)$ -- минимальный из размеров вершинных покрытий

$G-ab, ab\in E(G)$ -- граф, полученный из $G$ удалением ребра $ab$

$G/ab, ab\in E(G)$ -- граф, полученный из $G$ стягиванием ребра $ab$

$G-v, v \in V(G)$ -- граф, полученный из $G$ удалением вершины $v$

$\VCk=\left\{G \mid \tau(G) \le k\right\}$

$\AVCk = \left\{G \mid \forall e \in E(G): (G - e) \in \VCk\right\}$

$\VCFk$ и $\AVCFk$ -- множества запрещённых миноров для $\VCk$ и $\AVCk$ соответственно.

$\CVCFk = \left\{G \in \VCFk \mid G - \text{связен}\right\}$

Поясним выбор некоторых названий: VC значит Vertex Covered, AVC значит Almost Vertex Covered, суффикс F - Forbidden, а префикс c - connected.

Мы имеем право говорить о множествах запрещённых миноров для свойств $\VCk$ и $\AVCk$ ввиду следующей леммы:

\begin{lemma}

Свойства $\VCk$ и $\AVCk$ замкнуты относительно взятия миноров

\end{lemma}

\begin{proof}

Сначала докажем для $\VCk$. Пусть $G \in \VCk$ (есть покрытие из $\le k$ вершин), а $H$ получен путём

\begin{itemize}
    \item Удаления ребра: тогда достаточно взять то же самое вершинное покрытие

    \item Удаления вершины: пересекаем старое покрытие с новым множеством вершин -- размер покрытия не увеличится, а все рёбра, инцидентные новым вершинам, мы из $H$ выкидываем

    \item Стягиванием ребра: если стягиваемое ребро было инцидентно вершине из покрытия, то добавляем новую вершину в покрытие, иначе оставляем покрытие как есть. Очевидно, полученное таким образом покрытие будет покрытием $H$.
\end{itemize}

В замкнутости относительно взятия миноров свойства $\AVCk$ также нетрудно убедиться.

\end{proof}

Знание таких списков запрещённых миноров важно, потому что позволяет создавать сертификаты не-принадлежности к заданному семейству: достаточно указать присутствующий запрещённый минор из списка и последовательность действий, которая позволяет его получить.

В этой работе были получены списки $\AVCFk$ и $\VCFk$ для $k \le 6$. В то время как $\VCFk$ при тех же $k$ -- уже известный результат (\cite{dinneen-xiong}), нам не удалось найти опубликованный список $\AVCFk$.

В ходе этой работы сначала были получены списки связных запрещённых миноров для $\VCk$, потом были получены $\VCFk$, потом на их основании были с помощью добавления рёбер были получены $\AVCFk$.

Размеры найденных множеств приведены в таблице \ref{tab:sz}.

\begin{table}[ht]
\caption{Размеры найденных множеств}
\begin{center}
\begin{tabular}{c|c|c|c}
$k$ & $|\CVCFk|$ & $|\VCFk|$ & $|\AVCFk|$\\
\hline
0 & 1 & 1 & 2\\
1 & 1 & 2 & 3\\
2 & 2 & 4 & 5\\
3 & 3 & 8 & 12\\
4 & 8 & 18 & 36\\
5 & 31 & 56 & 174\\
6 & 188 & 260 & 1478\\
\end{tabular}
\end{center}
\label{tab:sz}
\end{table}

С текстом исходного кода и с полученными списками запрещённых миноров можно ознакомиться в репозитории \cite{repo}. Найденные графы для $k=4$ также приведены в заключении.

\section{Поиск $\CVCFk$}

Как мы увидим позже, все элементы $\VCFk$ довольно просто строятся из связных компонент, поэтому оправданно сначала найти связные компоненты отдельно.

Наша общая стратегия для поиска связных запрещённых миноров будет такой: сначала получим факты о виде графов из $\CVCFk$, которые позволят ограничить перебор, а потом переберём оставшиеся графы и в лоб проверим, подходят они или нет. Факты, как и сама процедура поиска $\VCFk$, заимствованы из \cite{dinneen-xiong} и \cite{dinneen-lai2007}.

Начнём с того, что построим следующий удобный критерий $\VCFk$:

\begin{theorem}
Пусть $k\ge 0$ и $G$ --- граф без изолированных вершин.
Тогда следующие условия эквивалентны:

\begin{enumerate}
  \item $G\in \VCFk$;
  \item $\tau(G)=k+1$ и для любого ребра $e\in E(G)$ выполнено $\tau(G-e)=k$.
\end{enumerate}

\end{theorem}

\begin{proof}

Докажем $(1)\Rightarrow(2)$.
Пусть $G\in \VCFk$.
Тогда $\tau(G)>k$. Покажем, что на самом деле $\tau(G)=k+1$.
Предположим противное: $\tau(G)\ge k+2$. Возьмём любую вершину $v\in V(G)$.
Тогда для графа $G-v$ верно
\[
\tau(G-v)\ \ge\ \tau(G)-1\ \ge\ k+1,
\]
поскольку любое вершинное покрытие графа $G-v$ можно расширить до покрытия $G$,
добавив вершину $v$ (то есть $\tau(G)\le \tau(G-v)+1$).
Следовательно, $\tau(G-v)>k$, то есть строгий минор $G-v$ не принадлежит $VC_k$,
что противоречит минимальности $G$ по минорам. Значит $\tau(G)=k+1$.

Теперь возьмём произвольное ребро $e\in E(G)$. Граф $G-e$ является строгим минором $G$,
поэтому по определению $\VCFk$ имеем $\tau(G-e)\le k$.
С другой стороны, всегда верно неравенство
\[
\tau(G)\ \le\ \tau(G-e)+1,
\]
потому что любое вершинное покрытие $S$ графа $G-e$ можно превратить
в вершинное покрытие графа $G$, добавив один из концов ребра $e$.
Отсюда при $\tau(G)=k+1$ получаем $\tau(G-e)\ge k$.

Докажем $(2)\Rightarrow(1)$.
Пусть $\tau(G)=k+1$ и $\tau(G-e)=k$ для всех $e\in E(G)$.
Тогда $\tau(G)>k$, то есть $G\notin \VCk$.
Осталось показать минор-минимальность: любой строгий минор $H <_m G$ лежит в $\VCk$.

\begin{enumerate}

\item Удаление ребра.
По условию для любого ребра $e$ уже имеем $\tau(G-e)=k$, то есть $G-e\in \VCk$.

\item Удаление вершины.
Пусть $v\in V(G)$. Поскольку в $G$ нет изолированных вершин, существует ребро $e$,
инцидентное $v$. Тогда $G-v$ является минором графа $G-e$ (достаточно удалить вершину $v$
после удаления ребра $e$), то есть $G-v \le_m G-e$.
Так как $\tau$ не возрастает при переходе к минорам, получаем
\[
\tau(G-v)\ \le\ \tau(G-e)\ =\ k.
\]
Значит $G-v\in \VCk$.

\item{Стягивание ребра.}
Пусть $e \in E(G)$ и рассмотрим $G/e$.
Заметим, что $\tau (G/e) \le \tau(G - e) = k$, ведь вершинное покрытие $G - e$ можно перенести на вершинное покрытие $G/e$, т.е. $G/e \in \VCk$

\end{enumerate}

Получили, что строгие миноры $G$ лежат в $\VCk$, то есть $G \in \VCFk$

\end{proof}

Теперь докажем, что графы из $\CVCFk$ двусвязны, то есть не имеют точек сочленения

\begin{lemma}

Любой связный запрещённый минор $G$ двусвязен.

\end{lemma}

\begin{proof}

Пусть $G \in \CVCFk$, $v$ - точка сочленения $G$. Обозначим как $C_i$ компоненты, на которые распадается $G - v$, и как $C'_i$ индуцированные подграфы на $C_i \cup \{v\}$. $\bigcup C_i$ -- собственный минор $G$ полученный путём удаления $v$, а значит $\sum \tau(C_i) = k$. При этом $\sum \tau(C'_i) \ge \tau(G) = k+1$, а значит для некоторого $i$ выполнено $\tau(C'_i) = \tau(C_i)+1$. Получается, что $\sum\limits_{j \ne i} \tau(C_j) + \tau(C_i) = k+1$, а это собственный минор -- противоречие.

\end{proof}

Также приведём без доказательства два важных утверждения, которые помогают ограничить поиск запрещённых миноров. За доказательством этих утверждений можно обратиться к \cite{dinneen-lai2007} и \cite{dinneen-xiong}.

\begin{proposition}

Количество вершин графа из $\CVCFk$ не превосходит $2k+1$.

\end{proposition}

\begin{proposition}

Степени вершин графов из $\CVCFk$ не превосходят $2k-n+3$, где $n$ - количество вершин.

\end{proposition}

Опишем теперь на псевдокоде алгоритм, который использовался для поиска $\CVCFk$:

\begin{lstlisting}
def is_vcf(G):
    if tau(G) <= k:
        return false

    for e in E(G):
        if not tau(G - e) <= k:
            return false

    return true

result = []

for n in {k+1, ..., 2k+1}:
    for G in BiconGraphs(n, d=2k-n+3):
        if is_vcf(G):
            result.append(G)
\end{lstlisting}

Здесь $\text{BiconGraphs}(n, d)$ - это двусвязные графы на $n$ вершинах, где степень не превосходит $d$. 

Детали реализации: код написан на Python с использованием библиотеки networkx \cite{networkx}. Перебор неизоморфных двусвязных графов с ограничением на степень был использован готовый из пакета nauty \cite{nauty}. Проверки вида $\tau(G) \le k$ осуществляются так: мы перебором/поиском в глубину рекурсивно пытаемся выбрать один из концов ребра в подпокрытие. Дополнительно используется наблюдение о том, что если $\text{deg} (v) > k$, то $v$ точно будет в покрытии.

\section{Вычисление несвязных запрещённых миноров}

Начнём с леммы, которая описывает, как выглядят несвязные запрещённые миноры.

\begin{lemma}
Пусть $O \in \VCFk$ и $O$ несвязен. Тогда $O = \bigsqcup\limits_{j=1}^s O_j$,
где каждый $O_j$ --- связные запрещённые миноры для некоторого $p_j$,
то есть $O_j\in \mathrm{cVCF}_{p_j}$, и выполняется баланс:
\[
\sum_{j=1}^s (p_j+1)=k+1.
\]
\end{lemma}

\begin{proof}

Для несвязного графа вершинное покрытие аддитивно по компонентам:
$\tau(\bigsqcup_j G_j)=\sum_j \tau(G_j)$.

Так как $O\in\VCFk$, то $\tau(O)=k+1$ и удаление любого ребра уменьшает $\tau$ на $1$.
Выбирая ребро внутри компоненты $O_j$, видим, что именно в этой компоненте покрытие уменьшается на $1$,
а остальные компоненты не меняются. Значит каждая компонента $O_j$ сама является запрещённым минором некоторого уровня $p_j=\tau(O_j)-1$.

Суммируя $\tau(O)=\sum_j \tau(O_j)=k+1$, получаем требуемое равенство
$\sum_j(p_j+1)=k + 1$.

\end{proof}

Таким образом, чтобы найти несвязные запрещённые миноры, нужно перебрать все разбиения $k+1$ на слагаемые, и для каждого слагаемого в разбиении подобрать уже известный связный запрещённый минор.

Отметим, что по итогу этого вычисления найденное количество миноров сошлось с результатами \cite{dinneen-xiong}.

\section{Вычисление $\AVCFk$}

На высоком уровне алгоритм, вычисляющий $\AVCFk$, работает так: сначала создаётся список кандидатов, полученных всевозможными добавлениями рёбер к элементам $\VCFk$, потом фильтруются неизоморфные, потом в лоб проверяется принадлежность $\AVCFk$.

Формализуем и обоснуем процедуру добавки рёбер к минорам из $\VCFk$ следующей теоремой.

\begin{theorem}

Для любого графа $H$ из $\AVCFk$ существует граф $G$ из $\VCFk$ такой, что $H$ можно получить из $G$ выполнением одной из трёх операций: соединение пары вершин ребром, добавление ребра, прикреплённого к $G$, или добавление не пересекающегося с $G$ ребра.

\end{theorem}

\begin{proof}

Пусть $H\in \AVCFk$. Тогда $H\notin\AVCk$, значит существует ребро $e$ такое, что $\tau(H-e) \ge k+1$.
Обозначим $G=H-e$. Так как $G$ -- собственный минор $H$, по минимальности $H$ имеем $G\in\AVCk$.
В частности, для всех рёбер $f\in E(G)$ выполнено $\tau(G-f)\le k$.
Пусть $G'$ -- это $G$, из которого выкинули все изолированные вершины. Для $G'$ выполнено, что $\tau(G') = \tau(G) \ge k+1$. Однако $G' <_m H$, значит $G' \in \AVCk$, и после удаления любого ребра появится покрытие $k$ вершинами. $\tau(G') = k+1$, а значит по критерию $\VCFk$ получаем $G' \in \VCFk$.

Граф $H$ мог быть получен из $G'$ путём добавления изолированных вершин и ребра, то есть одним из трёх указанных действий.

\end{proof}

Отметим, что это не критерий: например, $K_3\sqcup K_2 \in \mathrm{VCF}_2$, но если соединить эти две компоненты ребром, то полученный граф не будет в $\mathrm{AVCF_2}$, ведь после удаления ребра из треугольника останется $P_4 \in \mathrm{VC}_2$.

Для проверки $G \in \AVCk$ перебирались все рёбра и проверялось $G-e\in \VCk$. Проверка $G \in \AVCFk$ выражалась через проверки $G \notin \AVCk$ и $\forall H<_mG: H \in \AVCk$.

Для фильтрации неизоморфных графов использовался тот же пакет nauty \cite{nauty}.

\section{Вывод}

В ходе проекта реализован вычислительный пайплайн для списков запрещённых миноров $\CVCFk$, $\VCFk$ и $\AVCFk$. На практике он применим для вычислений для небольших $k$ ввиду экспоненциально растущих временных затрат. Пайплайн был применён экспериментально для вычислений запрещённых миноров для $k \le 6$. Материалы для воспроизведения можно найти в репозитории \cite{repo}. Ниже приведена диаграмма с найденными запрещёнными минорами для $k=4$.

\begin{figure}[h]
\centering
\resizebox{\textwidth}{!}{\begin{tikzpicture}[x=1cm,y=1cm]
\tikzset{avnode/.style={circle,draw=black,inner sep=0.8pt}, avedge/.style={draw=black,line width=0.28pt}}
\begin{scope}[shift={(0.0000,-0.0000)}]
\node[avnode] (g00_n0) at (-0.2455,0.4200) {};
\node[avnode] (g00_n1) at (-0.1153,0.1726) {};
\node[avnode] (g00_n2) at (0.1177,-0.4200) {};
\node[avnode] (g00_n3) at (0.0247,-0.3245) {};
\node[avnode] (g00_n4) at (0.2455,-0.3535) {};
\node[avnode] (g00_n5) at (0.2176,-0.2223) {};
\node[avnode] (g00_n6) at (0.0513,-0.1414) {};
\draw[avedge] (g00_n0) -- (g00_n1);
\draw[avedge] (g00_n1) -- (g00_n6);
\draw[avedge] (g00_n2) -- (g00_n3);
\draw[avedge] (g00_n2) -- (g00_n4);
\draw[avedge] (g00_n2) -- (g00_n5);
\draw[avedge] (g00_n2) -- (g00_n6);
\draw[avedge] (g00_n3) -- (g00_n4);
\draw[avedge] (g00_n3) -- (g00_n5);
\draw[avedge] (g00_n3) -- (g00_n6);
\draw[avedge] (g00_n4) -- (g00_n5);
\draw[avedge] (g00_n4) -- (g00_n6);
\draw[avedge] (g00_n5) -- (g00_n6);
\end{scope}
\begin{scope}[shift={(2.1500,-0.0000)}]
\node[avnode] (g01_n0) at (0.3675,-0.2070) {};
\node[avnode] (g01_n1) at (0.4200,-0.0455) {};
\node[avnode] (g01_n2) at (-0.3584,0.0073) {};
\node[avnode] (g01_n3) at (-0.2937,0.2070) {};
\node[avnode] (g01_n4) at (-0.4200,0.1375) {};
\node[avnode] (g01_n5) at (0.1719,-0.0543) {};
\node[avnode] (g01_n6) at (-0.1550,0.0518) {};
\draw[avedge] (g01_n0) -- (g01_n1);
\draw[avedge] (g01_n0) -- (g01_n5);
\draw[avedge] (g01_n1) -- (g01_n5);
\draw[avedge] (g01_n2) -- (g01_n3);
\draw[avedge] (g01_n2) -- (g01_n4);
\draw[avedge] (g01_n2) -- (g01_n6);
\draw[avedge] (g01_n3) -- (g01_n4);
\draw[avedge] (g01_n3) -- (g01_n6);
\draw[avedge] (g01_n4) -- (g01_n6);
\draw[avedge] (g01_n5) -- (g01_n6);
\end{scope}
\begin{scope}[shift={(4.3000,-0.0000)}]
\node[avnode] (g02_n0) at (0.1861,0.4200) {};
\node[avnode] (g02_n1) at (-0.2239,0.3852) {};
\node[avnode] (g02_n2) at (0.1761,-0.3979) {};
\node[avnode] (g02_n3) at (-0.0729,-0.4200) {};
\node[avnode] (g02_n4) at (0.2239,-0.0485) {};
\node[avnode] (g02_n5) at (-0.1792,-0.0833) {};
\node[avnode] (g02_n6) at (-0.0036,0.2212) {};
\draw[avedge] (g02_n0) -- (g02_n1);
\draw[avedge] (g02_n0) -- (g02_n4);
\draw[avedge] (g02_n0) -- (g02_n6);
\draw[avedge] (g02_n1) -- (g02_n5);
\draw[avedge] (g02_n1) -- (g02_n6);
\draw[avedge] (g02_n2) -- (g02_n3);
\draw[avedge] (g02_n2) -- (g02_n4);
\draw[avedge] (g02_n2) -- (g02_n5);
\draw[avedge] (g02_n3) -- (g02_n4);
\draw[avedge] (g02_n3) -- (g02_n5);
\draw[avedge] (g02_n4) -- (g02_n6);
\draw[avedge] (g02_n5) -- (g02_n6);
\end{scope}
\begin{scope}[shift={(6.4500,-0.0000)}]
\node[avnode] (g03_n0) at (0.4200,-0.1366) {};
\node[avnode] (g03_n1) at (0.2401,0.2165) {};
\node[avnode] (g03_n2) at (-0.0306,-0.2165) {};
\node[avnode] (g03_n3) at (-0.3111,-0.2102) {};
\node[avnode] (g03_n4) at (-0.4200,-0.0133) {};
\node[avnode] (g03_n5) at (-0.0576,0.1150) {};
\node[avnode] (g03_n6) at (-0.2113,0.1272) {};
\draw[avedge] (g03_n0) -- (g03_n1);
\draw[avedge] (g03_n0) -- (g03_n2);
\draw[avedge] (g03_n1) -- (g03_n5);
\draw[avedge] (g03_n1) -- (g03_n6);
\draw[avedge] (g03_n2) -- (g03_n3);
\draw[avedge] (g03_n2) -- (g03_n4);
\draw[avedge] (g03_n2) -- (g03_n6);
\draw[avedge] (g03_n3) -- (g03_n4);
\draw[avedge] (g03_n3) -- (g03_n5);
\draw[avedge] (g03_n3) -- (g03_n6);
\draw[avedge] (g03_n4) -- (g03_n5);
\draw[avedge] (g03_n4) -- (g03_n6);
\draw[avedge] (g03_n5) -- (g03_n6);
\end{scope}
\begin{scope}[shift={(8.6000,-0.0000)}]
\node[avnode] (g04_n0) at (-0.4200,0.0600) {};
\node[avnode] (g04_n1) at (-0.0244,0.4056) {};
\node[avnode] (g04_n2) at (-0.2927,-0.4056) {};
\node[avnode] (g04_n3) at (0.4200,0.2297) {};
\node[avnode] (g04_n4) at (-0.1356,-0.1706) {};
\node[avnode] (g04_n5) at (0.1710,0.0995) {};
\node[avnode] (g04_n6) at (0.2015,-0.2400) {};
\draw[avedge] (g04_n0) -- (g04_n1);
\draw[avedge] (g04_n0) -- (g04_n2);
\draw[avedge] (g04_n0) -- (g04_n4);
\draw[avedge] (g04_n1) -- (g04_n3);
\draw[avedge] (g04_n1) -- (g04_n5);
\draw[avedge] (g04_n2) -- (g04_n4);
\draw[avedge] (g04_n2) -- (g04_n6);
\draw[avedge] (g04_n3) -- (g04_n5);
\draw[avedge] (g04_n3) -- (g04_n6);
\draw[avedge] (g04_n4) -- (g04_n5);
\draw[avedge] (g04_n4) -- (g04_n6);
\draw[avedge] (g04_n5) -- (g04_n6);
\end{scope}
\begin{scope}[shift={(10.7500,-0.0000)}]
\node[avnode] (g05_n0) at (0.2727,0.4020) {};
\node[avnode] (g05_n1) at (-0.0218,-0.4200) {};
\node[avnode] (g05_n2) at (-0.2727,0.2672) {};
\node[avnode] (g05_n3) at (-0.2081,0.0975) {};
\node[avnode] (g05_n4) at (0.2525,0.0574) {};
\node[avnode] (g05_n5) at (-0.0417,-0.0477) {};
\node[avnode] (g05_n6) at (0.0412,-0.1942) {};
\node[avnode] (g05_n7) at (-0.0625,0.4200) {};
\draw[avedge] (g05_n0) -- (g05_n4);
\draw[avedge] (g05_n0) -- (g05_n7);
\draw[avedge] (g05_n1) -- (g05_n5);
\draw[avedge] (g05_n1) -- (g05_n6);
\draw[avedge] (g05_n2) -- (g05_n3);
\draw[avedge] (g05_n2) -- (g05_n5);
\draw[avedge] (g05_n2) -- (g05_n7);
\draw[avedge] (g05_n3) -- (g05_n6);
\draw[avedge] (g05_n3) -- (g05_n7);
\draw[avedge] (g05_n4) -- (g05_n5);
\draw[avedge] (g05_n4) -- (g05_n6);
\end{scope}
\begin{scope}[shift={(0.0000,-2.1500)}]
\node[avnode] (g06_n0) at (-0.4200,-0.0925) {};
\node[avnode] (g06_n1) at (0.4200,0.3247) {};
\node[avnode] (g06_n2) at (0.1036,-0.1496) {};
\node[avnode] (g06_n3) at (0.2425,-0.3247) {};
\node[avnode] (g06_n4) at (-0.1053,0.0442) {};
\node[avnode] (g06_n5) at (0.3330,0.0223) {};
\node[avnode] (g06_n6) at (0.1255,0.1300) {};
\node[avnode] (g06_n7) at (-0.1138,-0.2231) {};
\draw[avedge] (g06_n0) -- (g06_n4);
\draw[avedge] (g06_n0) -- (g06_n7);
\draw[avedge] (g06_n1) -- (g06_n5);
\draw[avedge] (g06_n1) -- (g06_n6);
\draw[avedge] (g06_n2) -- (g06_n3);
\draw[avedge] (g06_n2) -- (g06_n4);
\draw[avedge] (g06_n2) -- (g06_n6);
\draw[avedge] (g06_n3) -- (g06_n5);
\draw[avedge] (g06_n3) -- (g06_n7);
\draw[avedge] (g06_n4) -- (g06_n5);
\draw[avedge] (g06_n6) -- (g06_n7);
\end{scope}
\begin{scope}[shift={(2.1500,-2.1500)}]
\node[avnode] (g07_n0) at (-0.1141,0.3936) {};
\node[avnode] (g07_n1) at (0.4200,0.0862) {};
\node[avnode] (g07_n2) at (-0.3323,-0.1602) {};
\node[avnode] (g07_n3) at (0.0293,-0.3936) {};
\node[avnode] (g07_n4) at (-0.4200,0.2124) {};
\node[avnode] (g07_n5) at (-0.0807,0.1115) {};
\node[avnode] (g07_n6) at (0.3983,-0.2813) {};
\node[avnode] (g07_n7) at (0.1411,-0.0535) {};
\draw[avedge] (g07_n0) -- (g07_n4);
\draw[avedge] (g07_n0) -- (g07_n7);
\draw[avedge] (g07_n1) -- (g07_n5);
\draw[avedge] (g07_n1) -- (g07_n6);
\draw[avedge] (g07_n2) -- (g07_n3);
\draw[avedge] (g07_n2) -- (g07_n4);
\draw[avedge] (g07_n2) -- (g07_n5);
\draw[avedge] (g07_n3) -- (g07_n6);
\draw[avedge] (g07_n3) -- (g07_n7);
\draw[avedge] (g07_n4) -- (g07_n5);
\draw[avedge] (g07_n6) -- (g07_n7);
\end{scope}
\begin{scope}[shift={(4.3000,-2.1500)}]
\node[avnode] (g08_n0) at (-0.1324,0.2315) {};
\node[avnode] (g08_n1) at (0.0841,-0.4200) {};
\node[avnode] (g08_n2) at (-0.1609,-0.3893) {};
\node[avnode] (g08_n3) at (0.0277,0.2345) {};
\node[avnode] (g08_n4) at (-0.0920,-0.0591) {};
\node[avnode] (g08_n5) at (0.0704,0.4200) {};
\node[avnode] (g08_n6) at (0.0988,-0.0880) {};
\node[avnode] (g08_n7) at (0.1609,0.1373) {};
\draw[avedge] (g08_n0) -- (g08_n5);
\draw[avedge] (g08_n0) -- (g08_n6);
\draw[avedge] (g08_n1) -- (g08_n2);
\draw[avedge] (g08_n1) -- (g08_n6);
\draw[avedge] (g08_n2) -- (g08_n4);
\draw[avedge] (g08_n3) -- (g08_n4);
\draw[avedge] (g08_n3) -- (g08_n5);
\draw[avedge] (g08_n3) -- (g08_n7);
\draw[avedge] (g08_n4) -- (g08_n7);
\draw[avedge] (g08_n5) -- (g08_n7);
\draw[avedge] (g08_n6) -- (g08_n7);
\end{scope}
\begin{scope}[shift={(6.4500,-2.1500)}]
\node[avnode] (g09_n0) at (0.2335,0.2223) {};
\node[avnode] (g09_n1) at (-0.3900,0.1002) {};
\node[avnode] (g09_n2) at (-0.0300,0.1623) {};
\node[avnode] (g09_n3) at (0.0447,0.2523) {};
\node[avnode] (g09_n4) at (-0.3264,0.2285) {};
\node[avnode] (g09_n5) at (-0.3007,0.3900) {};
\node[avnode] (g09_n6) at (-0.3900,0.3214) {};
\node[avnode] (g09_n7) at (-0.1869,0.2891) {};
\draw[avedge] (g09_n0) -- (g09_n3);
\draw[avedge] (g09_n1) -- (g09_n2);
\draw[avedge] (g09_n3) -- (g09_n7);
\draw[avedge] (g09_n4) -- (g09_n5);
\draw[avedge] (g09_n4) -- (g09_n6);
\draw[avedge] (g09_n4) -- (g09_n7);
\draw[avedge] (g09_n5) -- (g09_n6);
\draw[avedge] (g09_n5) -- (g09_n7);
\draw[avedge] (g09_n6) -- (g09_n7);
\end{scope}
\begin{scope}[shift={(8.6000,-2.1500)}]
\node[avnode] (g10_n0) at (-0.1718,-0.4200) {};
\node[avnode] (g10_n1) at (0.0065,0.4200) {};
\node[avnode] (g10_n2) at (0.2750,0.3137) {};
\node[avnode] (g10_n3) at (-0.2750,-0.1420) {};
\node[avnode] (g10_n4) at (0.0926,-0.2864) {};
\node[avnode] (g10_n5) at (-0.1903,0.1568) {};
\node[avnode] (g10_n6) at (0.2379,-0.0090) {};
\node[avnode] (g10_n7) at (-0.0337,-0.0658) {};
\draw[avedge] (g10_n0) -- (g10_n3);
\draw[avedge] (g10_n0) -- (g10_n4);
\draw[avedge] (g10_n1) -- (g10_n2);
\draw[avedge] (g10_n1) -- (g10_n5);
\draw[avedge] (g10_n2) -- (g10_n6);
\draw[avedge] (g10_n3) -- (g10_n5);
\draw[avedge] (g10_n3) -- (g10_n7);
\draw[avedge] (g10_n4) -- (g10_n6);
\draw[avedge] (g10_n4) -- (g10_n7);
\draw[avedge] (g10_n5) -- (g10_n7);
\draw[avedge] (g10_n6) -- (g10_n7);
\end{scope}
\begin{scope}[shift={(10.7500,-2.1500)}]
\node[avnode] (g11_n0) at (-0.3900,0.3900) {};
\node[avnode] (g11_n1) at (-0.3292,-0.1191) {};
\node[avnode] (g11_n2) at (-0.3703,0.2251) {};
\node[avnode] (g11_n3) at (-0.3489,0.0457) {};
\node[avnode] (g11_n4) at (0.2399,-0.0233) {};
\node[avnode] (g11_n5) at (-0.2692,0.2947) {};
\node[avnode] (g11_n6) at (-0.1735,-0.1187) {};
\node[avnode] (g11_n7) at (0.1442,0.3900) {};
\draw[avedge] (g11_n0) -- (g11_n2);
\draw[avedge] (g11_n1) -- (g11_n3);
\draw[avedge] (g11_n2) -- (g11_n3);
\draw[avedge] (g11_n4) -- (g11_n5);
\draw[avedge] (g11_n4) -- (g11_n6);
\draw[avedge] (g11_n4) -- (g11_n7);
\draw[avedge] (g11_n5) -- (g11_n6);
\draw[avedge] (g11_n5) -- (g11_n7);
\draw[avedge] (g11_n6) -- (g11_n7);
\end{scope}
\begin{scope}[shift={(0.0000,-4.3000)}]
\node[avnode] (g12_n0) at (0.2392,-0.0509) {};
\node[avnode] (g12_n1) at (0.2872,0.3900) {};
\node[avnode] (g12_n2) at (0.2632,0.1690) {};
\node[avnode] (g12_n3) at (-0.1356,0.3900) {};
\node[avnode] (g12_n4) at (-0.2318,-0.1757) {};
\node[avnode] (g12_n5) at (0.1792,0.2261) {};
\node[avnode] (g12_n6) at (0.1192,-0.1240) {};
\node[avnode] (g12_n7) at (-0.3900,0.1424) {};
\draw[avedge] (g12_n0) -- (g12_n2);
\draw[avedge] (g12_n1) -- (g12_n2);
\draw[avedge] (g12_n3) -- (g12_n4);
\draw[avedge] (g12_n3) -- (g12_n5);
\draw[avedge] (g12_n3) -- (g12_n6);
\draw[avedge] (g12_n3) -- (g12_n7);
\draw[avedge] (g12_n4) -- (g12_n5);
\draw[avedge] (g12_n4) -- (g12_n6);
\draw[avedge] (g12_n4) -- (g12_n7);
\draw[avedge] (g12_n5) -- (g12_n6);
\draw[avedge] (g12_n5) -- (g12_n7);
\draw[avedge] (g12_n6) -- (g12_n7);
\end{scope}
\begin{scope}[shift={(2.1500,-4.3000)}]
\node[avnode] (g13_n0) at (0.0872,0.3951) {};
\node[avnode] (g13_n1) at (0.4200,-0.0820) {};
\node[avnode] (g13_n2) at (0.4179,0.2733) {};
\node[avnode] (g13_n3) at (-0.4200,-0.1268) {};
\node[avnode] (g13_n4) at (-0.2319,-0.3951) {};
\node[avnode] (g13_n5) at (-0.2260,0.1622) {};
\node[avnode] (g13_n6) at (0.1021,-0.3061) {};
\node[avnode] (g13_n7) at (-0.1846,-0.1613) {};
\draw[avedge] (g13_n0) -- (g13_n2);
\draw[avedge] (g13_n0) -- (g13_n5);
\draw[avedge] (g13_n1) -- (g13_n2);
\draw[avedge] (g13_n1) -- (g13_n6);
\draw[avedge] (g13_n3) -- (g13_n4);
\draw[avedge] (g13_n3) -- (g13_n5);
\draw[avedge] (g13_n3) -- (g13_n7);
\draw[avedge] (g13_n4) -- (g13_n6);
\draw[avedge] (g13_n4) -- (g13_n7);
\draw[avedge] (g13_n5) -- (g13_n7);
\draw[avedge] (g13_n6) -- (g13_n7);
\end{scope}
\begin{scope}[shift={(4.3000,-4.3000)}]
\node[avnode] (g14_n0) at (-0.0263,-0.4200) {};
\node[avnode] (g14_n1) at (-0.2737,-0.3872) {};
\node[avnode] (g14_n2) at (0.1220,0.4200) {};
\node[avnode] (g14_n3) at (0.2369,-0.0496) {};
\node[avnode] (g14_n4) at (-0.0556,0.1756) {};
\node[avnode] (g14_n5) at (0.2737,0.1838) {};
\node[avnode] (g14_n6) at (-0.1230,-0.1049) {};
\node[avnode] (g14_n7) at (0.0922,-0.0855) {};
\draw[avedge] (g14_n0) -- (g14_n1);
\draw[avedge] (g14_n0) -- (g14_n7);
\draw[avedge] (g14_n1) -- (g14_n6);
\draw[avedge] (g14_n2) -- (g14_n4);
\draw[avedge] (g14_n2) -- (g14_n5);
\draw[avedge] (g14_n3) -- (g14_n5);
\draw[avedge] (g14_n3) -- (g14_n6);
\draw[avedge] (g14_n3) -- (g14_n7);
\draw[avedge] (g14_n4) -- (g14_n6);
\draw[avedge] (g14_n4) -- (g14_n7);
\draw[avedge] (g14_n5) -- (g14_n7);
\end{scope}
\begin{scope}[shift={(6.4500,-4.3000)}]
\node[avnode] (g15_n0) at (-0.3900,-0.2509) {};
\node[avnode] (g15_n1) at (-0.2068,-0.2271) {};
\node[avnode] (g15_n2) at (0.1792,-0.2454) {};
\node[avnode] (g15_n3) at (0.1617,-0.1109) {};
\node[avnode] (g15_n4) at (-0.3900,0.3900) {};
\node[avnode] (g15_n5) at (-0.3398,-0.0509) {};
\node[avnode] (g15_n6) at (0.0169,0.2130) {};
\node[avnode] (g15_n7) at (0.0100,-0.1990) {};
\draw[avedge] (g15_n0) -- (g15_n1);
\draw[avedge] (g15_n1) -- (g15_n7);
\draw[avedge] (g15_n2) -- (g15_n3);
\draw[avedge] (g15_n2) -- (g15_n7);
\draw[avedge] (g15_n3) -- (g15_n7);
\draw[avedge] (g15_n4) -- (g15_n5);
\draw[avedge] (g15_n4) -- (g15_n6);
\draw[avedge] (g15_n5) -- (g15_n6);
\end{scope}
\begin{scope}[shift={(8.6000,-4.3000)}]
\node[avnode] (g16_n0) at (-0.4200,-0.2343) {};
\node[avnode] (g16_n1) at (-0.2110,-0.1466) {};
\node[avnode] (g16_n2) at (0.1625,0.1982) {};
\node[avnode] (g16_n3) at (0.3808,0.2343) {};
\node[avnode] (g16_n4) at (0.2517,-0.0378) {};
\node[avnode] (g16_n5) at (0.2781,-0.1511) {};
\node[avnode] (g16_n6) at (0.4200,0.0179) {};
\node[avnode] (g16_n7) at (0.0577,-0.0424) {};
\draw[avedge] (g16_n0) -- (g16_n1);
\draw[avedge] (g16_n1) -- (g16_n7);
\draw[avedge] (g16_n2) -- (g16_n3);
\draw[avedge] (g16_n2) -- (g16_n7);
\draw[avedge] (g16_n3) -- (g16_n6);
\draw[avedge] (g16_n4) -- (g16_n5);
\draw[avedge] (g16_n4) -- (g16_n6);
\draw[avedge] (g16_n4) -- (g16_n7);
\draw[avedge] (g16_n5) -- (g16_n6);
\draw[avedge] (g16_n5) -- (g16_n7);
\end{scope}
\begin{scope}[shift={(10.7500,-4.3000)}]
\node[avnode] (g17_n0) at (-0.3900,0.3900) {};
\node[avnode] (g17_n1) at (-0.1361,0.0300) {};
\node[avnode] (g17_n2) at (-0.3564,-0.0300) {};
\node[avnode] (g17_n3) at (-0.3900,-0.1642) {};
\node[avnode] (g17_n4) at (0.2001,-0.3114) {};
\node[avnode] (g17_n5) at (0.2335,-0.1779) {};
\node[avnode] (g17_n6) at (-0.2007,-0.1403) {};
\node[avnode] (g17_n7) at (0.0448,-0.2017) {};
\draw[avedge] (g17_n0) -- (g17_n1);
\draw[avedge] (g17_n2) -- (g17_n3);
\draw[avedge] (g17_n2) -- (g17_n6);
\draw[avedge] (g17_n3) -- (g17_n6);
\draw[avedge] (g17_n4) -- (g17_n5);
\draw[avedge] (g17_n4) -- (g17_n7);
\draw[avedge] (g17_n5) -- (g17_n7);
\draw[avedge] (g17_n6) -- (g17_n7);
\end{scope}
\begin{scope}[shift={(0.0000,-6.4500)}]
\node[avnode] (g18_n0) at (0.1811,0.3900) {};
\node[avnode] (g18_n1) at (0.2674,0.0300) {};
\node[avnode] (g18_n2) at (0.0338,0.3900) {};
\node[avnode] (g18_n3) at (-0.2873,0.2537) {};
\node[avnode] (g18_n4) at (0.1211,0.0524) {};
\node[avnode] (g18_n5) at (-0.3900,-0.0820) {};
\node[avnode] (g18_n6) at (-0.0816,-0.2335) {};
\node[avnode] (g18_n7) at (-0.1589,-0.0016) {};
\draw[avedge] (g18_n0) -- (g18_n1);
\draw[avedge] (g18_n2) -- (g18_n3);
\draw[avedge] (g18_n2) -- (g18_n4);
\draw[avedge] (g18_n3) -- (g18_n5);
\draw[avedge] (g18_n3) -- (g18_n7);
\draw[avedge] (g18_n4) -- (g18_n6);
\draw[avedge] (g18_n4) -- (g18_n7);
\draw[avedge] (g18_n5) -- (g18_n6);
\draw[avedge] (g18_n5) -- (g18_n7);
\draw[avedge] (g18_n6) -- (g18_n7);
\end{scope}
\begin{scope}[shift={(2.1500,-6.4500)}]
\node[avnode] (g19_n0) at (0.4200,0.2741) {};
\node[avnode] (g19_n1) at (0.2420,0.1492) {};
\node[avnode] (g19_n2) at (-0.3616,-0.2741) {};
\node[avnode] (g19_n3) at (-0.4200,-0.0938) {};
\node[avnode] (g19_n4) at (-0.1422,-0.2165) {};
\node[avnode] (g19_n5) at (-0.2137,0.0002) {};
\node[avnode] (g19_n6) at (-0.0719,-0.1033) {};
\node[avnode] (g19_n7) at (0.0127,-0.0023) {};
\draw[avedge] (g19_n0) -- (g19_n1);
\draw[avedge] (g19_n1) -- (g19_n7);
\draw[avedge] (g19_n2) -- (g19_n3);
\draw[avedge] (g19_n2) -- (g19_n4);
\draw[avedge] (g19_n3) -- (g19_n5);
\draw[avedge] (g19_n4) -- (g19_n6);
\draw[avedge] (g19_n4) -- (g19_n7);
\draw[avedge] (g19_n5) -- (g19_n6);
\draw[avedge] (g19_n5) -- (g19_n7);
\draw[avedge] (g19_n6) -- (g19_n7);
\end{scope}
\begin{scope}[shift={(4.3000,-6.4500)}]
\node[avnode] (g20_n0) at (0.4200,0.1588) {};
\node[avnode] (g20_n1) at (0.3255,0.3196) {};
\node[avnode] (g20_n2) at (-0.3386,-0.3196) {};
\node[avnode] (g20_n3) at (-0.4200,-0.2073) {};
\node[avnode] (g20_n4) at (0.2544,0.0239) {};
\node[avnode] (g20_n5) at (0.1287,0.2349) {};
\node[avnode] (g20_n6) at (-0.2216,-0.1486) {};
\node[avnode] (g20_n7) at (0.0166,0.0195) {};
\draw[avedge] (g20_n0) -- (g20_n1);
\draw[avedge] (g20_n0) -- (g20_n4);
\draw[avedge] (g20_n1) -- (g20_n5);
\draw[avedge] (g20_n2) -- (g20_n3);
\draw[avedge] (g20_n2) -- (g20_n6);
\draw[avedge] (g20_n3) -- (g20_n6);
\draw[avedge] (g20_n4) -- (g20_n7);
\draw[avedge] (g20_n5) -- (g20_n7);
\draw[avedge] (g20_n6) -- (g20_n7);
\end{scope}
\begin{scope}[shift={(6.4500,-6.4500)}]
\node[avnode] (g21_n0) at (0.4200,0.2992) {};
\node[avnode] (g21_n1) at (0.2236,0.1908) {};
\node[avnode] (g21_n2) at (-0.2569,0.0841) {};
\node[avnode] (g21_n3) at (-0.0183,0.0534) {};
\node[avnode] (g21_n4) at (-0.4200,-0.0869) {};
\node[avnode] (g21_n5) at (-0.3566,-0.2992) {};
\node[avnode] (g21_n6) at (-0.3228,-0.2031) {};
\node[avnode] (g21_n7) at (-0.1537,-0.1759) {};
\draw[avedge] (g21_n0) -- (g21_n1);
\draw[avedge] (g21_n1) -- (g21_n3);
\draw[avedge] (g21_n2) -- (g21_n3);
\draw[avedge] (g21_n2) -- (g21_n4);
\draw[avedge] (g21_n3) -- (g21_n7);
\draw[avedge] (g21_n4) -- (g21_n5);
\draw[avedge] (g21_n4) -- (g21_n6);
\draw[avedge] (g21_n5) -- (g21_n6);
\draw[avedge] (g21_n5) -- (g21_n7);
\draw[avedge] (g21_n6) -- (g21_n7);
\end{scope}
\begin{scope}[shift={(8.6000,-6.4500)}]
\node[avnode] (g22_n0) at (-0.3900,0.3900) {};
\node[avnode] (g22_n1) at (-0.0689,-0.0900) {};
\node[avnode] (g22_n2) at (-0.3900,-0.3567) {};
\node[avnode] (g22_n3) at (-0.0506,-0.1500) {};
\node[avnode] (g22_n4) at (-0.0089,0.1577) {};
\node[avnode] (g22_n5) at (0.3652,-0.0257) {};
\node[avnode] (g22_n6) at (0.3370,0.3900) {};
\node[avnode] (g22_n7) at (-0.2865,0.2353) {};
\node[avnode] (g22_n8) at (-0.1724,0.0647) {};
\draw[avedge] (g22_n0) -- (g22_n7);
\draw[avedge] (g22_n1) -- (g22_n8);
\draw[avedge] (g22_n2) -- (g22_n3);
\draw[avedge] (g22_n4) -- (g22_n5);
\draw[avedge] (g22_n4) -- (g22_n6);
\draw[avedge] (g22_n5) -- (g22_n6);
\draw[avedge] (g22_n7) -- (g22_n8);
\end{scope}
\begin{scope}[shift={(10.7500,-6.4500)}]
\node[avnode] (g23_n0) at (0.3829,0.3900) {};
\node[avnode] (g23_n1) at (0.1993,-0.0900) {};
\node[avnode] (g23_n2) at (-0.3459,-0.0880) {};
\node[avnode] (g23_n3) at (-0.0165,-0.1467) {};
\node[avnode] (g23_n4) at (-0.3900,0.2435) {};
\node[avnode] (g23_n5) at (0.1393,0.1493) {};
\node[avnode] (g23_n6) at (-0.0909,0.3900) {};
\node[avnode] (g23_n7) at (0.3237,0.2352) {};
\node[avnode] (g23_n8) at (0.2586,0.0651) {};
\draw[avedge] (g23_n0) -- (g23_n7);
\draw[avedge] (g23_n1) -- (g23_n8);
\draw[avedge] (g23_n2) -- (g23_n3);
\draw[avedge] (g23_n2) -- (g23_n4);
\draw[avedge] (g23_n3) -- (g23_n5);
\draw[avedge] (g23_n4) -- (g23_n6);
\draw[avedge] (g23_n5) -- (g23_n6);
\draw[avedge] (g23_n7) -- (g23_n8);
\end{scope}
\begin{scope}[shift={(0.0000,-8.6000)}]
\node[avnode] (g24_n0) at (0.1683,-0.3145) {};
\node[avnode] (g24_n1) at (-0.1683,-0.4200) {};
\node[avnode] (g24_n2) at (-0.1154,0.4200) {};
\node[avnode] (g24_n3) at (0.1497,0.2058) {};
\node[avnode] (g24_n4) at (-0.1683,0.0833) {};
\node[avnode] (g24_n5) at (0.1683,0.0166) {};
\node[avnode] (g24_n6) at (-0.0157,-0.2659) {};
\node[avnode] (g24_n7) at (-0.1683,0.0347) {};
\node[avnode] (g24_n8) at (-0.0004,-0.3674) {};
\draw[avedge] (g24_n0) -- (g24_n8);
\draw[avedge] (g24_n1) -- (g24_n8);
\draw[avedge] (g24_n2) -- (g24_n3);
\draw[avedge] (g24_n2) -- (g24_n4);
\draw[avedge] (g24_n3) -- (g24_n4);
\draw[avedge] (g24_n5) -- (g24_n6);
\draw[avedge] (g24_n5) -- (g24_n7);
\draw[avedge] (g24_n6) -- (g24_n7);
\end{scope}
\begin{scope}[shift={(2.1500,-8.6000)}]
\node[avnode] (g25_n0) at (-0.3900,-0.4133) {};
\node[avnode] (g25_n1) at (0.0257,-0.1500) {};
\node[avnode] (g25_n2) at (0.1490,0.3900) {};
\node[avnode] (g25_n3) at (0.1659,0.0506) {};
\node[avnode] (g25_n4) at (-0.1826,-0.2819) {};
\node[avnode] (g25_n5) at (-0.2320,0.3900) {};
\node[avnode] (g25_n6) at (0.0890,0.2313) {};
\node[avnode] (g25_n7) at (-0.0690,-0.0900) {};
\node[avnode] (g25_n8) at (-0.3900,0.0687) {};
\draw[avedge] (g25_n0) -- (g25_n4);
\draw[avedge] (g25_n1) -- (g25_n4);
\draw[avedge] (g25_n2) -- (g25_n3);
\draw[avedge] (g25_n5) -- (g25_n6);
\draw[avedge] (g25_n5) -- (g25_n7);
\draw[avedge] (g25_n5) -- (g25_n8);
\draw[avedge] (g25_n6) -- (g25_n7);
\draw[avedge] (g25_n6) -- (g25_n8);
\draw[avedge] (g25_n7) -- (g25_n8);
\end{scope}
\begin{scope}[shift={(4.3000,-8.6000)}]
\node[avnode] (g26_n0) at (-0.2713,-0.4200) {};
\node[avnode] (g26_n1) at (0.1347,-0.2126) {};
\node[avnode] (g26_n2) at (0.0526,-0.1541) {};
\node[avnode] (g26_n3) at (0.2713,0.0441) {};
\node[avnode] (g26_n4) at (-0.0682,-0.3163) {};
\node[avnode] (g26_n5) at (-0.1448,0.4200) {};
\node[avnode] (g26_n6) at (-0.2713,0.3008) {};
\node[avnode] (g26_n7) at (-0.1879,0.0790) {};
\node[avnode] (g26_n8) at (0.0693,0.3118) {};
\draw[avedge] (g26_n0) -- (g26_n4);
\draw[avedge] (g26_n1) -- (g26_n4);
\draw[avedge] (g26_n2) -- (g26_n3);
\draw[avedge] (g26_n2) -- (g26_n7);
\draw[avedge] (g26_n3) -- (g26_n8);
\draw[avedge] (g26_n5) -- (g26_n6);
\draw[avedge] (g26_n5) -- (g26_n7);
\draw[avedge] (g26_n5) -- (g26_n8);
\draw[avedge] (g26_n6) -- (g26_n7);
\draw[avedge] (g26_n6) -- (g26_n8);
\end{scope}
\begin{scope}[shift={(6.4500,-8.6000)}]
\node[avnode] (g27_n0) at (-0.3900,-0.0602) {};
\node[avnode] (g27_n1) at (-0.0506,0.3900) {};
\node[avnode] (g27_n2) at (-0.3900,0.1635) {};
\node[avnode] (g27_n3) at (0.0702,0.1035) {};
\node[avnode] (g27_n4) at (0.1016,-0.0877) {};
\node[avnode] (g27_n5) at (-0.2367,-0.0399) {};
\node[avnode] (g27_n6) at (0.2231,0.1006) {};
\node[avnode] (g27_n7) at (0.2450,-0.0359) {};
\node[avnode] (g27_n8) at (-0.0512,-0.0138) {};
\draw[avedge] (g27_n0) -- (g27_n5);
\draw[avedge] (g27_n1) -- (g27_n2);
\draw[avedge] (g27_n3) -- (g27_n6);
\draw[avedge] (g27_n3) -- (g27_n8);
\draw[avedge] (g27_n4) -- (g27_n7);
\draw[avedge] (g27_n4) -- (g27_n8);
\draw[avedge] (g27_n5) -- (g27_n8);
\draw[avedge] (g27_n6) -- (g27_n7);
\end{scope}
\begin{scope}[shift={(8.6000,-8.6000)}]
\node[avnode] (g28_n0) at (0.2254,-0.4200) {};
\node[avnode] (g28_n1) at (0.1577,-0.2629) {};
\node[avnode] (g28_n2) at (-0.1137,0.0167) {};
\node[avnode] (g28_n3) at (0.1460,0.1253) {};
\node[avnode] (g28_n4) at (-0.1963,0.3530) {};
\node[avnode] (g28_n5) at (-0.0354,0.4200) {};
\node[avnode] (g28_n6) at (-0.2254,0.1748) {};
\node[avnode] (g28_n7) at (0.1116,0.3160) {};
\node[avnode] (g28_n8) at (0.0738,-0.0661) {};
\draw[avedge] (g28_n0) -- (g28_n1);
\draw[avedge] (g28_n1) -- (g28_n8);
\draw[avedge] (g28_n2) -- (g28_n6);
\draw[avedge] (g28_n2) -- (g28_n8);
\draw[avedge] (g28_n3) -- (g28_n7);
\draw[avedge] (g28_n3) -- (g28_n8);
\draw[avedge] (g28_n4) -- (g28_n5);
\draw[avedge] (g28_n4) -- (g28_n6);
\draw[avedge] (g28_n5) -- (g28_n7);
\end{scope}
\begin{scope}[shift={(10.7500,-8.6000)}]
\node[avnode] (g29_n0) at (-0.1899,0.3900) {};
\node[avnode] (g29_n1) at (-0.1604,0.0506) {};
\node[avnode] (g29_n2) at (-0.1004,0.3900) {};
\node[avnode] (g29_n3) at (0.2348,0.0506) {};
\node[avnode] (g29_n4) at (-0.2882,0.3900) {};
\node[avnode] (g29_n5) at (-0.2499,0.2225) {};
\node[avnode] (g29_n6) at (-0.2579,-0.1467) {};
\node[avnode] (g29_n7) at (-0.3900,-0.1110) {};
\node[avnode] (g29_n8) at (-0.2853,0.0203) {};
\draw[avedge] (g29_n0) -- (g29_n1);
\draw[avedge] (g29_n2) -- (g29_n3);
\draw[avedge] (g29_n4) -- (g29_n5);
\draw[avedge] (g29_n5) -- (g29_n8);
\draw[avedge] (g29_n6) -- (g29_n7);
\draw[avedge] (g29_n6) -- (g29_n8);
\draw[avedge] (g29_n7) -- (g29_n8);
\end{scope}
\begin{scope}[shift={(0.0000,-10.7500)}]
\node[avnode] (g30_n0) at (-0.3900,-0.1647) {};
\node[avnode] (g30_n1) at (0.0044,-0.2273) {};
\node[avnode] (g30_n2) at (0.1791,0.3900) {};
\node[avnode] (g30_n3) at (0.3244,0.0680) {};
\node[avnode] (g30_n4) at (-0.3131,-0.0858) {};
\node[avnode] (g30_n5) at (-0.3900,0.2194) {};
\node[avnode] (g30_n6) at (0.0011,-0.1047) {};
\node[avnode] (g30_n7) at (-0.1231,0.3900) {};
\node[avnode] (g30_n8) at (0.1191,0.1893) {};
\node[avnode] (g30_n9) at (-0.1924,-0.1961) {};
\draw[avedge] (g30_n0) -- (g30_n9);
\draw[avedge] (g30_n1) -- (g30_n9);
\draw[avedge] (g30_n2) -- (g30_n3);
\draw[avedge] (g30_n4) -- (g30_n5);
\draw[avedge] (g30_n4) -- (g30_n6);
\draw[avedge] (g30_n5) -- (g30_n7);
\draw[avedge] (g30_n6) -- (g30_n8);
\draw[avedge] (g30_n7) -- (g30_n8);
\end{scope}
\begin{scope}[shift={(2.1500,-10.7500)}]
\node[avnode] (g31_n0) at (-0.0563,-0.4200) {};
\node[avnode] (g31_n1) at (-0.2346,-0.1065) {};
\node[avnode] (g31_n2) at (-0.1414,-0.0080) {};
\node[avnode] (g31_n3) at (0.0667,-0.0588) {};
\node[avnode] (g31_n4) at (-0.2346,0.1841) {};
\node[avnode] (g31_n5) at (0.2329,0.2851) {};
\node[avnode] (g31_n6) at (0.0665,0.4200) {};
\node[avnode] (g31_n7) at (0.2346,0.0714) {};
\node[avnode] (g31_n8) at (-0.1429,0.3761) {};
\node[avnode] (g31_n9) at (-0.1457,-0.2629) {};
\draw[avedge] (g31_n0) -- (g31_n9);
\draw[avedge] (g31_n1) -- (g31_n9);
\draw[avedge] (g31_n2) -- (g31_n3);
\draw[avedge] (g31_n2) -- (g31_n4);
\draw[avedge] (g31_n3) -- (g31_n7);
\draw[avedge] (g31_n4) -- (g31_n8);
\draw[avedge] (g31_n5) -- (g31_n6);
\draw[avedge] (g31_n5) -- (g31_n7);
\draw[avedge] (g31_n6) -- (g31_n8);
\end{scope}
\begin{scope}[shift={(4.3000,-10.7500)}]
\node[avnode] (g32_n0) at (-0.3900,-0.0644) {};
\node[avnode] (g32_n1) at (-0.1143,-0.3864) {};
\node[avnode] (g32_n2) at (0.2677,-0.2909) {};
\node[avnode] (g32_n3) at (-0.0543,-0.0644) {};
\node[avnode] (g32_n4) at (-0.3900,0.3900) {};
\node[avnode] (g32_n5) at (-0.2697,-0.0044) {};
\node[avnode] (g32_n6) at (-0.3300,0.1932) {};
\node[avnode] (g32_n7) at (0.1847,0.0550) {};
\node[avnode] (g32_n8) at (-0.2097,0.0419) {};
\node[avnode] (g32_n9) at (-0.0238,0.3900) {};
\draw[avedge] (g32_n0) -- (g32_n1);
\draw[avedge] (g32_n2) -- (g32_n3);
\draw[avedge] (g32_n4) -- (g32_n6);
\draw[avedge] (g32_n5) -- (g32_n6);
\draw[avedge] (g32_n7) -- (g32_n8);
\draw[avedge] (g32_n7) -- (g32_n9);
\draw[avedge] (g32_n8) -- (g32_n9);
\end{scope}
\begin{scope}[shift={(6.4500,-10.7500)}]
\node[avnode] (g33_n0) at (-0.0015,0.3900) {};
\node[avnode] (g33_n1) at (0.1169,0.0680) {};
\node[avnode] (g33_n2) at (0.3555,0.0680) {};
\node[avnode] (g33_n3) at (0.1769,0.3900) {};
\node[avnode] (g33_n4) at (-0.3900,-0.1944) {};
\node[avnode] (g33_n5) at (-0.0680,-0.1254) {};
\node[avnode] (g33_n6) at (-0.3900,0.3900) {};
\node[avnode] (g33_n7) at (-0.0615,-0.0654) {};
\node[avnode] (g33_n8) at (-0.2222,0.3078) {};
\node[avnode] (g33_n9) at (-0.1313,0.1186) {};
\draw[avedge] (g33_n0) -- (g33_n1);
\draw[avedge] (g33_n2) -- (g33_n3);
\draw[avedge] (g33_n4) -- (g33_n5);
\draw[avedge] (g33_n6) -- (g33_n8);
\draw[avedge] (g33_n7) -- (g33_n9);
\draw[avedge] (g33_n8) -- (g33_n9);
\end{scope}
\begin{scope}[shift={(8.6000,-10.7500)}]
\node[avnode] (g34_n0) at (-0.0140,0.0100) {};
\node[avnode] (g34_n1) at (-0.3900,-0.2735) {};
\node[avnode] (g34_n2) at (0.2071,0.0805) {};
\node[avnode] (g34_n3) at (-0.1129,0.3900) {};
\node[avnode] (g34_n4) at (0.0460,0.0100) {};
\node[avnode] (g34_n5) at (0.3660,-0.1065) {};
\node[avnode] (g34_n6) at (-0.3608,0.0700) {};
\node[avnode] (g34_n7) at (-0.3900,0.3900) {};
\node[avnode] (g34_n8) at (-0.1729,0.0700) {};
\node[avnode] (g34_n9) at (-0.3008,0.3900) {};
\node[avnode] (g34_n10) at (-0.2016,-0.1314) {};
\draw[avedge] (g34_n0) -- (g34_n10);
\draw[avedge] (g34_n1) -- (g34_n10);
\draw[avedge] (g34_n2) -- (g34_n3);
\draw[avedge] (g34_n4) -- (g34_n5);
\draw[avedge] (g34_n6) -- (g34_n7);
\draw[avedge] (g34_n8) -- (g34_n9);
\end{scope}
\begin{scope}[shift={(10.7500,-10.7500)}]
\node[avnode] (g35_n0) at (-0.3900,0.3900) {};
\node[avnode] (g35_n1) at (-0.2339,0.0700) {};
\node[avnode] (g35_n2) at (0.3236,0.3900) {};
\node[avnode] (g35_n3) at (0.0036,0.2193) {};
\node[avnode] (g35_n4) at (-0.0700,-0.2353) {};
\node[avnode] (g35_n5) at (-0.3900,-0.2048) {};
\node[avnode] (g35_n6) at (-0.1739,0.3900) {};
\node[avnode] (g35_n7) at (-0.0564,0.0700) {};
\node[avnode] (g35_n8) at (-0.0100,-0.0240) {};
\node[avnode] (g35_n9) at (0.3100,0.0100) {};
\node[avnode] (g35_n10) at (-0.3900,0.0100) {};
\node[avnode] (g35_n11) at (-0.0700,-0.1448) {};
\draw[avedge] (g35_n0) -- (g35_n1);
\draw[avedge] (g35_n2) -- (g35_n3);
\draw[avedge] (g35_n4) -- (g35_n5);
\draw[avedge] (g35_n6) -- (g35_n7);
\draw[avedge] (g35_n8) -- (g35_n9);
\draw[avedge] (g35_n10) -- (g35_n11);
\end{scope}
\end{tikzpicture}}
\caption{Все графы из $\mathrm{AVCF}_4$}
\end{figure}

\begin{figure}[h]
\centering
\resizebox{\textwidth}{!}{\begin{tikzpicture}[x=1cm,y=1cm]
\tikzset{avnode/.style={circle,draw=black,inner sep=0.8pt}, avedge/.style={draw=black,line width=0.28pt}}
\begin{scope}[shift={(0.0000,-0.0000)}]
\node[avnode] (g00_n0) at (-0.0276,0.4200) {};
\node[avnode] (g00_n1) at (-0.3933,0.1592) {};
\node[avnode] (g00_n2) at (0.0396,-0.4200) {};
\node[avnode] (g00_n3) at (-0.1994,-0.1468) {};
\node[avnode] (g00_n4) at (0.3933,-0.1457) {};
\node[avnode] (g00_n5) at (0.3534,0.2641) {};
\draw[avedge] (g00_n0) -- (g00_n1);
\draw[avedge] (g00_n0) -- (g00_n2);
\draw[avedge] (g00_n0) -- (g00_n3);
\draw[avedge] (g00_n0) -- (g00_n4);
\draw[avedge] (g00_n0) -- (g00_n5);
\draw[avedge] (g00_n1) -- (g00_n2);
\draw[avedge] (g00_n1) -- (g00_n3);
\draw[avedge] (g00_n1) -- (g00_n4);
\draw[avedge] (g00_n1) -- (g00_n5);
\draw[avedge] (g00_n2) -- (g00_n3);
\draw[avedge] (g00_n2) -- (g00_n4);
\draw[avedge] (g00_n2) -- (g00_n5);
\draw[avedge] (g00_n3) -- (g00_n4);
\draw[avedge] (g00_n3) -- (g00_n5);
\draw[avedge] (g00_n4) -- (g00_n5);
\end{scope}
\begin{scope}[shift={(2.1500,-0.0000)}]
\node[avnode] (g01_n0) at (0.1337,-0.2874) {};
\node[avnode] (g01_n1) at (0.1949,0.4165) {};
\node[avnode] (g01_n2) at (-0.1345,-0.4200) {};
\node[avnode] (g01_n3) at (-0.0344,0.4200) {};
\node[avnode] (g01_n4) at (-0.3010,-0.2629) {};
\node[avnode] (g01_n5) at (0.3010,0.1387) {};
\node[avnode] (g01_n6) at (-0.1414,0.0683) {};
\draw[avedge] (g01_n0) -- (g01_n2);
\draw[avedge] (g01_n0) -- (g01_n4);
\draw[avedge] (g01_n0) -- (g01_n5);
\draw[avedge] (g01_n1) -- (g01_n3);
\draw[avedge] (g01_n1) -- (g01_n5);
\draw[avedge] (g01_n1) -- (g01_n6);
\draw[avedge] (g01_n2) -- (g01_n4);
\draw[avedge] (g01_n2) -- (g01_n6);
\draw[avedge] (g01_n3) -- (g01_n5);
\draw[avedge] (g01_n3) -- (g01_n6);
\draw[avedge] (g01_n4) -- (g01_n6);
\end{scope}
\begin{scope}[shift={(4.3000,-0.0000)}]
\node[avnode] (g02_n0) at (0.0389,0.3182) {};
\node[avnode] (g02_n1) at (0.0679,-0.3753) {};
\node[avnode] (g02_n2) at (0.2505,0.3753) {};
\node[avnode] (g02_n3) at (-0.4200,-0.2657) {};
\node[avnode] (g02_n4) at (0.1843,-0.0210) {};
\node[avnode] (g02_n5) at (-0.2972,0.2322) {};
\node[avnode] (g02_n6) at (0.4200,0.0166) {};
\draw[avedge] (g02_n0) -- (g02_n2);
\draw[avedge] (g02_n0) -- (g02_n4);
\draw[avedge] (g02_n0) -- (g02_n5);
\draw[avedge] (g02_n0) -- (g02_n6);
\draw[avedge] (g02_n1) -- (g02_n3);
\draw[avedge] (g02_n1) -- (g02_n4);
\draw[avedge] (g02_n1) -- (g02_n6);
\draw[avedge] (g02_n2) -- (g02_n4);
\draw[avedge] (g02_n2) -- (g02_n5);
\draw[avedge] (g02_n2) -- (g02_n6);
\draw[avedge] (g02_n3) -- (g02_n5);
\draw[avedge] (g02_n4) -- (g02_n6);
\end{scope}
\begin{scope}[shift={(6.4500,-0.0000)}]
\node[avnode] (g03_n0) at (-0.2803,-0.2059) {};
\node[avnode] (g03_n1) at (0.3820,0.2059) {};
\node[avnode] (g03_n2) at (-0.4200,-0.0624) {};
\node[avnode] (g03_n3) at (0.4200,-0.1495) {};
\node[avnode] (g03_n4) at (-0.0682,0.1410) {};
\node[avnode] (g03_n5) at (-0.0319,-0.1853) {};
\node[avnode] (g03_n6) at (-0.3154,0.1104) {};
\draw[avedge] (g03_n0) -- (g03_n2);
\draw[avedge] (g03_n0) -- (g03_n4);
\draw[avedge] (g03_n0) -- (g03_n5);
\draw[avedge] (g03_n0) -- (g03_n6);
\draw[avedge] (g03_n1) -- (g03_n3);
\draw[avedge] (g03_n1) -- (g03_n4);
\draw[avedge] (g03_n2) -- (g03_n4);
\draw[avedge] (g03_n2) -- (g03_n5);
\draw[avedge] (g03_n2) -- (g03_n6);
\draw[avedge] (g03_n3) -- (g03_n5);
\draw[avedge] (g03_n4) -- (g03_n6);
\draw[avedge] (g03_n5) -- (g03_n6);
\end{scope}
\begin{scope}[shift={(8.6000,-0.0000)}]
\node[avnode] (g04_n0) at (0.2150,0.3641) {};
\node[avnode] (g04_n1) at (-0.2150,-0.3580) {};
\node[avnode] (g04_n2) at (-0.0745,0.1352) {};
\node[avnode] (g04_n3) at (0.1397,0.0828) {};
\node[avnode] (g04_n4) at (0.0034,-0.4200) {};
\node[avnode] (g04_n5) at (-0.0058,0.4200) {};
\node[avnode] (g04_n6) at (-0.1312,-0.0781) {};
\node[avnode] (g04_n7) at (0.0794,-0.1382) {};
\draw[avedge] (g04_n0) -- (g04_n3);
\draw[avedge] (g04_n0) -- (g04_n5);
\draw[avedge] (g04_n1) -- (g04_n4);
\draw[avedge] (g04_n1) -- (g04_n6);
\draw[avedge] (g04_n2) -- (g04_n5);
\draw[avedge] (g04_n2) -- (g04_n6);
\draw[avedge] (g04_n2) -- (g04_n7);
\draw[avedge] (g04_n3) -- (g04_n6);
\draw[avedge] (g04_n3) -- (g04_n7);
\draw[avedge] (g04_n4) -- (g04_n7);
\end{scope}
\begin{scope}[shift={(10.7500,-0.0000)}]
\node[avnode] (g05_n0) at (0.1973,-0.2470) {};
\node[avnode] (g05_n1) at (0.2036,0.2268) {};
\node[avnode] (g05_n2) at (-0.4138,0.1646) {};
\node[avnode] (g05_n3) at (-0.1428,-0.3331) {};
\node[avnode] (g05_n4) at (0.2723,-0.0157) {};
\node[avnode] (g05_n5) at (-0.1353,0.3331) {};
\node[avnode] (g05_n6) at (-0.4200,-0.1517) {};
\node[avnode] (g05_n7) at (0.4200,-0.0072) {};
\draw[avedge] (g05_n0) -- (g05_n3);
\draw[avedge] (g05_n0) -- (g05_n4);
\draw[avedge] (g05_n0) -- (g05_n7);
\draw[avedge] (g05_n1) -- (g05_n4);
\draw[avedge] (g05_n1) -- (g05_n5);
\draw[avedge] (g05_n1) -- (g05_n7);
\draw[avedge] (g05_n2) -- (g05_n5);
\draw[avedge] (g05_n2) -- (g05_n6);
\draw[avedge] (g05_n3) -- (g05_n6);
\draw[avedge] (g05_n4) -- (g05_n7);
\end{scope}
\begin{scope}[shift={(0.0000,-2.1500)}]
\node[avnode] (g06_n0) at (-0.2989,-0.1083) {};
\node[avnode] (g06_n1) at (0.0529,0.4200) {};
\node[avnode] (g06_n2) at (0.2112,-0.0801) {};
\node[avnode] (g06_n3) at (-0.1937,-0.4200) {};
\node[avnode] (g06_n4) at (-0.2034,0.2118) {};
\node[avnode] (g06_n5) at (0.2989,0.2463) {};
\node[avnode] (g06_n6) at (0.1084,-0.3990) {};
\node[avnode] (g06_n7) at (-0.0815,0.0035) {};
\draw[avedge] (g06_n0) -- (g06_n3);
\draw[avedge] (g06_n0) -- (g06_n4);
\draw[avedge] (g06_n0) -- (g06_n7);
\draw[avedge] (g06_n1) -- (g06_n4);
\draw[avedge] (g06_n1) -- (g06_n5);
\draw[avedge] (g06_n2) -- (g06_n5);
\draw[avedge] (g06_n2) -- (g06_n6);
\draw[avedge] (g06_n2) -- (g06_n7);
\draw[avedge] (g06_n3) -- (g06_n6);
\draw[avedge] (g06_n4) -- (g06_n7);
\end{scope}
\begin{scope}[shift={(2.1500,-2.1500)}]
\node[avnode] (g07_n0) at (0.0413,0.4200) {};
\node[avnode] (g07_n1) at (0.4141,0.0221) {};
\node[avnode] (g07_n2) at (0.0898,-0.4200) {};
\node[avnode] (g07_n3) at (-0.3989,-0.1754) {};
\node[avnode] (g07_n4) at (-0.2423,0.3480) {};
\node[avnode] (g07_n5) at (0.3016,0.2902) {};
\node[avnode] (g07_n6) at (0.3311,-0.2573) {};
\node[avnode] (g07_n7) at (-0.1991,-0.3884) {};
\node[avnode] (g07_n8) at (-0.4141,0.1139) {};
\draw[avedge] (g07_n0) -- (g07_n4);
\draw[avedge] (g07_n0) -- (g07_n5);
\draw[avedge] (g07_n1) -- (g07_n5);
\draw[avedge] (g07_n1) -- (g07_n6);
\draw[avedge] (g07_n2) -- (g07_n6);
\draw[avedge] (g07_n2) -- (g07_n7);
\draw[avedge] (g07_n3) -- (g07_n7);
\draw[avedge] (g07_n3) -- (g07_n8);
\draw[avedge] (g07_n4) -- (g07_n8);
\end{scope}
\begin{scope}[shift={(4.3000,-2.1500)}]
\node[avnode] (g08_n0) at (-0.0587,0.4200) {};
\node[avnode] (g08_n1) at (0.2659,-0.0519) {};
\node[avnode] (g08_n2) at (-0.0735,-0.1517) {};
\node[avnode] (g08_n3) at (0.2746,0.3021) {};
\node[avnode] (g08_n4) at (-0.2746,0.1394) {};
\node[avnode] (g08_n5) at (-0.2746,-0.4200) {};
\node[avnode] (g08_n6) at (0.0870,-0.2081) {};
\draw[avedge] (g08_n0) -- (g08_n1);
\draw[avedge] (g08_n0) -- (g08_n2);
\draw[avedge] (g08_n0) -- (g08_n3);
\draw[avedge] (g08_n0) -- (g08_n4);
\draw[avedge] (g08_n1) -- (g08_n2);
\draw[avedge] (g08_n1) -- (g08_n3);
\draw[avedge] (g08_n1) -- (g08_n4);
\draw[avedge] (g08_n2) -- (g08_n3);
\draw[avedge] (g08_n2) -- (g08_n4);
\draw[avedge] (g08_n3) -- (g08_n4);
\draw[avedge] (g08_n5) -- (g08_n6);
\end{scope}
\begin{scope}[shift={(6.4500,-2.1500)}]
\node[avnode] (g09_n0) at (0.0600,0.0668) {};
\node[avnode] (g09_n1) at (-0.2674,0.2199) {};
\node[avnode] (g09_n2) at (-0.1142,-0.2335) {};
\node[avnode] (g09_n3) at (-0.0860,0.3900) {};
\node[avnode] (g09_n4) at (-0.3900,-0.1010) {};
\node[avnode] (g09_n5) at (0.0782,0.3124) {};
\node[avnode] (g09_n6) at (-0.3900,-0.3556) {};
\node[avnode] (g09_n7) at (-0.0300,-0.2935) {};
\draw[avedge] (g09_n0) -- (g09_n2);
\draw[avedge] (g09_n0) -- (g09_n3);
\draw[avedge] (g09_n0) -- (g09_n5);
\draw[avedge] (g09_n1) -- (g09_n3);
\draw[avedge] (g09_n1) -- (g09_n4);
\draw[avedge] (g09_n1) -- (g09_n5);
\draw[avedge] (g09_n2) -- (g09_n4);
\draw[avedge] (g09_n3) -- (g09_n5);
\draw[avedge] (g09_n6) -- (g09_n7);
\end{scope}
\begin{scope}[shift={(8.6000,-2.1500)}]
\node[avnode] (g10_n0) at (-0.2202,0.2902) {};
\node[avnode] (g10_n1) at (0.2450,0.2989) {};
\node[avnode] (g10_n2) at (-0.0353,0.1804) {};
\node[avnode] (g10_n3) at (-0.3900,0.3851) {};
\node[avnode] (g10_n4) at (0.0339,0.3731) {};
\node[avnode] (g10_n5) at (0.1830,0.1213) {};
\node[avnode] (g10_n6) at (-0.2054,0.3900) {};
\node[avnode] (g10_n7) at (-0.3900,-0.0306) {};
\node[avnode] (g10_n8) at (-0.0506,0.0613) {};
\draw[avedge] (g10_n0) -- (g10_n3);
\draw[avedge] (g10_n0) -- (g10_n4);
\draw[avedge] (g10_n1) -- (g10_n4);
\draw[avedge] (g10_n1) -- (g10_n5);
\draw[avedge] (g10_n2) -- (g10_n5);
\draw[avedge] (g10_n2) -- (g10_n6);
\draw[avedge] (g10_n3) -- (g10_n6);
\draw[avedge] (g10_n7) -- (g10_n8);
\end{scope}
\begin{scope}[shift={(10.7500,-2.1500)}]
\node[avnode] (g11_n0) at (0.1589,0.4200) {};
\node[avnode] (g11_n1) at (-0.1589,-0.0050) {};
\node[avnode] (g11_n2) at (-0.2122,0.3661) {};
\node[avnode] (g11_n3) at (0.2122,0.0488) {};
\node[avnode] (g11_n4) at (0.1097,-0.4200) {};
\node[avnode] (g11_n5) at (-0.2122,-0.2413) {};
\node[avnode] (g11_n6) at (0.1035,-0.0519) {};
\draw[avedge] (g11_n0) -- (g11_n1);
\draw[avedge] (g11_n0) -- (g11_n2);
\draw[avedge] (g11_n0) -- (g11_n3);
\draw[avedge] (g11_n1) -- (g11_n2);
\draw[avedge] (g11_n1) -- (g11_n3);
\draw[avedge] (g11_n2) -- (g11_n3);
\draw[avedge] (g11_n4) -- (g11_n5);
\draw[avedge] (g11_n4) -- (g11_n6);
\draw[avedge] (g11_n5) -- (g11_n6);
\end{scope}
\begin{scope}[shift={(0.0000,-4.3000)}]
\node[avnode] (g12_n0) at (-0.1517,-0.0094) {};
\node[avnode] (g12_n1) at (0.0069,0.4200) {};
\node[avnode] (g12_n2) at (0.1293,-0.0211) {};
\node[avnode] (g12_n3) at (-0.2266,0.2632) {};
\node[avnode] (g12_n4) at (0.2266,0.2443) {};
\node[avnode] (g12_n5) at (-0.2266,-0.0689) {};
\node[avnode] (g12_n6) at (-0.1713,-0.4200) {};
\node[avnode] (g12_n7) at (0.1051,-0.1966) {};
\draw[avedge] (g12_n0) -- (g12_n2);
\draw[avedge] (g12_n0) -- (g12_n3);
\draw[avedge] (g12_n1) -- (g12_n3);
\draw[avedge] (g12_n1) -- (g12_n4);
\draw[avedge] (g12_n2) -- (g12_n4);
\draw[avedge] (g12_n5) -- (g12_n6);
\draw[avedge] (g12_n5) -- (g12_n7);
\draw[avedge] (g12_n6) -- (g12_n7);
\end{scope}
\begin{scope}[shift={(2.1500,-4.3000)}]
\node[avnode] (g13_n0) at (0.1191,0.3541) {};
\node[avnode] (g13_n1) at (0.0796,-0.1152) {};
\node[avnode] (g13_n2) at (-0.3504,0.3900) {};
\node[avnode] (g13_n3) at (-0.3900,-0.0793) {};
\node[avnode] (g13_n4) at (-0.3900,-0.1966) {};
\node[avnode] (g13_n5) at (-0.0300,-0.1752) {};
\node[avnode] (g13_n6) at (0.1791,0.3900) {};
\node[avnode] (g13_n7) at (0.3415,0.0300) {};
\draw[avedge] (g13_n0) -- (g13_n1);
\draw[avedge] (g13_n0) -- (g13_n2);
\draw[avedge] (g13_n0) -- (g13_n3);
\draw[avedge] (g13_n1) -- (g13_n2);
\draw[avedge] (g13_n1) -- (g13_n3);
\draw[avedge] (g13_n2) -- (g13_n3);
\draw[avedge] (g13_n4) -- (g13_n5);
\draw[avedge] (g13_n6) -- (g13_n7);
\end{scope}
\begin{scope}[shift={(4.3000,-4.3000)}]
\node[avnode] (g14_n0) at (0.1346,0.1531) {};
\node[avnode] (g14_n1) at (-0.3473,0.0722) {};
\node[avnode] (g14_n2) at (-0.0061,0.4200) {};
\node[avnode] (g14_n3) at (-0.0788,-0.0614) {};
\node[avnode] (g14_n4) at (-0.3021,0.3712) {};
\node[avnode] (g14_n5) at (0.3473,0.1153) {};
\node[avnode] (g14_n6) at (0.1885,0.4200) {};
\node[avnode] (g14_n7) at (-0.0516,-0.1153) {};
\node[avnode] (g14_n8) at (-0.3473,-0.4200) {};
\draw[avedge] (g14_n0) -- (g14_n2);
\draw[avedge] (g14_n0) -- (g14_n3);
\draw[avedge] (g14_n1) -- (g14_n3);
\draw[avedge] (g14_n1) -- (g14_n4);
\draw[avedge] (g14_n2) -- (g14_n4);
\draw[avedge] (g14_n5) -- (g14_n6);
\draw[avedge] (g14_n7) -- (g14_n8);
\end{scope}
\begin{scope}[shift={(6.4500,-4.3000)}]
\node[avnode] (g15_n0) at (-0.0215,0.0734) {};
\node[avnode] (g15_n1) at (0.1525,-0.1784) {};
\node[avnode] (g15_n2) at (-0.1525,-0.2032) {};
\node[avnode] (g15_n3) at (-0.1525,0.4200) {};
\node[avnode] (g15_n4) at (-0.1178,0.1149) {};
\node[avnode] (g15_n5) at (0.1290,0.2975) {};
\node[avnode] (g15_n6) at (0.0966,-0.4200) {};
\node[avnode] (g15_n7) at (-0.1525,-0.2447) {};
\draw[avedge] (g15_n0) -- (g15_n1);
\draw[avedge] (g15_n0) -- (g15_n2);
\draw[avedge] (g15_n1) -- (g15_n2);
\draw[avedge] (g15_n3) -- (g15_n4);
\draw[avedge] (g15_n3) -- (g15_n5);
\draw[avedge] (g15_n4) -- (g15_n5);
\draw[avedge] (g15_n6) -- (g15_n7);
\end{scope}
\begin{scope}[shift={(8.6000,-4.3000)}]
\node[avnode] (g16_n0) at (-0.3900,0.3900) {};
\node[avnode] (g16_n1) at (-0.2774,-0.0250) {};
\node[avnode] (g16_n2) at (0.0257,0.2800) {};
\node[avnode] (g16_n3) at (0.0857,0.3900) {};
\node[avnode] (g16_n4) at (0.2105,0.0506) {};
\node[avnode] (g16_n5) at (-0.2017,-0.4244) {};
\node[avnode] (g16_n6) at (-0.3900,-0.0850) {};
\node[avnode] (g16_n7) at (0.1977,-0.0850) {};
\node[avnode] (g16_n8) at (-0.1417,-0.1577) {};
\draw[avedge] (g16_n0) -- (g16_n1);
\draw[avedge] (g16_n0) -- (g16_n2);
\draw[avedge] (g16_n1) -- (g16_n2);
\draw[avedge] (g16_n3) -- (g16_n4);
\draw[avedge] (g16_n5) -- (g16_n6);
\draw[avedge] (g16_n7) -- (g16_n8);
\end{scope}
\begin{scope}[shift={(10.7500,-4.3000)}]
\node[avnode] (g17_n0) at (-0.3900,0.3900) {};
\node[avnode] (g17_n1) at (-0.1629,0.0680) {};
\node[avnode] (g17_n2) at (0.3140,0.0080) {};
\node[avnode] (g17_n3) at (-0.0080,-0.0359) {};
\node[avnode] (g17_n4) at (0.3085,0.0785) {};
\node[avnode] (g17_n5) at (-0.0135,0.3900) {};
\node[avnode] (g17_n6) at (-0.3900,0.0080) {};
\node[avnode] (g17_n7) at (-0.0680,-0.1092) {};
\node[avnode] (g17_n8) at (-0.0735,0.0680) {};
\node[avnode] (g17_n9) at (-0.1029,0.3900) {};
\draw[avedge] (g17_n0) -- (g17_n1);
\draw[avedge] (g17_n2) -- (g17_n3);
\draw[avedge] (g17_n4) -- (g17_n5);
\draw[avedge] (g17_n6) -- (g17_n7);
\draw[avedge] (g17_n8) -- (g17_n9);
\end{scope}
\end{tikzpicture}}
\caption{Все графы из $\mathrm{VCF}_4$}
\end{figure}

\section*{Ссылки}
\begin{thebibliography}{9}

\bibitem{dinneen-lai2007}
M. J. Dinneen, R. Lai.
\newblock Properties of vertex cover obstructions.
\newblock \emph{Discrete Mathematics}
\url{https://www.cs.auckland.ac.nz/~mjd/vacs/VCproperties.pdf}

\bibitem{networkx}

Библиотека для работы с графами NetworkX

\newblock \url{https://networkx.org/en/}

\bibitem{dinneen-xiong}
M. J. Dinneen, L. Xiong.
\newblock The Minor-Order Obstructions for The Graphs of Vertex Cover Six
\newblock \emph{Discrete Mathematics}.

\url{https://www.cs.auckland.ac.nz/research/groups/CDMTCS/researchreports/118vc6.pdf}

\bibitem{robertson-seymour}
N. Robertson, P. D. Seymour.
\newblock Graph Minors.
\newblock \emph{Серия статей}, 1983--2004.

\bibitem{nauty}
McKay, B. D. and Piperno, A.
\newblock Practical Graph Isomorphism, II
\newblock Journal of Symbolic Computation, 60 (2014), pp. 94-112.

\bibitem{repo}

Репозиторий с текстом исходного кода и полученными списками запрещённых миноров

\url{https://github.com/koldun256/vertex_cover}

\end{thebibliography}

\end{document}
